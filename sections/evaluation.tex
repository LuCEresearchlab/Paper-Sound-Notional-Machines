\section{Evaluation}
\label{chr:Evaluation}

%We then
%evaluate the entire framework
%by comparing the notional machines that appeared
%in Section~\ref{chr:Modeling} and Section~\ref{chr:RevealingInconsistencies}
%to an existing dataset of \numOfNMs notional machines.
%
%We classify the notional machines in the dataset according to various dimensions and show that the notional machines we analyzed are representative of the design space of notional machines used in practice.

The notional machines we presented so far,
are meant not only to exemplify how to use our framework to reason about notional machines
but also were chosen to be representative of the design space of notional machines used in practice.
%
To characterize this design space we analyzed the notional machines in the dataset captured by~\citet{fincherNotionalMachinesComputing2020}
and classified them according to three dimensions.
For each dimension,
we present the values (the categories) of the dimension,
the count and percentage of notional machines that fall into each value,
and
whether at least one of the notional machines we analyzed falls into each value.

%
The first dimension is the form of the notional machine (Table~\ref{tab:nm-classification-form}),
which can be
metaphorical (primarely inspired by and represented with real world objects) or
diagrammatic (represented with diagrams).
The distinction between the two is not always clear-cut,
given that
a notional machine may mix both forms
and
some real world objects can be represented diagrammatically.
%
Nevertheless this distinction is useful because metaphorical notional machines may be more difficult to be made sound, given the existing degrees of freedom and constraints of the real world objects they are inspired by.


\begin{table}[]
\begin{tabular}{|l||r|l|l|}
\hline
\textbf{Form}  & \textbf{Count} & \textbf{Percentage} & \textbf{Covered} \\
\hline
\hline
Metaphorical        & 19    & 51.35\%    & yes \\ \hline
Diagramatic & 17    & 45.95\%    & yes \\ \hline
Both           & 1     & 2.70\%     & -   \\ \hline
\hline
\textbf{Total} & \numOfNMs    & 100.00\%   & -   \\
\hline
\end{tabular}
\caption{Notional machines in the dataset published by \citet{fincherNotionalMachinesComputing2020} classified according to their form.}
\label{tab:nm-classification-form}
\end{table}


The second dimension is the broad focus of the notional machine (Table~\ref{tab:nm-classification-broad-focus}).
We say \emph{broad} focus because,
although
most notional machines in the dataset focused on the runtime semantics of a programming language construct
(or set of constructs),
we wanted to distinguish these notional machines from others that focus on other aspects of programs,
such as type-checking or parsing.


\begin{table}[]
\begin{tabular}{|l||r|l|l|}
\hline
\textbf{Broad Focus} & \textbf{Count} & \textbf{Percentage} & \textbf{Covered} \\
\hline
\hline
Evaluation     & 30    & 70\%    & yes \\ \hline
Data Structure & 3     & xx\%    & yes \\ \hline
Type-checking  & 2     & xx\%    & yes \\ \hline
Parsing        & 2     & xx\%    & yes \\ \hline
??             & 2     & xx\%    & -   \\ \hline
Logic Gates    & 1     & xx\%    & no  \\ \hline
\hline
\textbf{Total} & \numOfNMs    & 100.00\%  & -   \\
\hline
\end{tabular}
\caption{Notional machines in the dataset published by \citet{fincherNotionalMachinesComputing2020} classified according to their broad focus.}
\label{tab:nm-classification-broad-focus}
\end{table}


The  third dimension
breaks down the notion of machines that focus on evaluation into the specific constructs they focus on (Table~\ref{tab:nm-classification-narrow-focus}).


\begin{table}[]
\begin{tabular}{|l||r|l|l|}
\hline
\textbf{Focused Construct} & \textbf{Count} & \textbf{Percentage} & \textbf{Covered} \\
\hline
\hline
Memory  &  7  &  23.33\%  & yes \\ \hline
Variables  &  4  &  13.33\%  & yes \\ \hline
Arrays  &  4  &  13.33\%  & yes \\ \hline
Functions  &  3  &  10.00\%  & yes \\ \hline
Method Calls  &  2  &  6.67\%  & no \\ \hline
Control Flow  &  2  &  6.67\%  & no \\ \hline
Call Stack  &  2  &  6.67\%  & no \\ \hline
Tracing  &  1  &  3.33\%  & no \\ \hline
String literals  &  1  &  3.33\%  & no \\ \hline
Procedure  &  1  &  3.33\%  & no \\ \hline
Objects  &  1  &  3.33\%  & no \\ \hline
Instructions  &  1  &  3.33\%  & no \\ \hline
Expression  &  1  &  3.33\%  & yes \\ \hline
%Memory          & 7     & 18.92\%   & yes \\ \hline
%Variables       & 5     & 13.51\%   & yes \\ \hline
%Arrays          & 4     & 10.81\%   & yes \\ \hline
%Functions       & 3     & 8.11\%    & yes \\ \hline
%Expression      & 3     & 8.11\%    & yes \\ \hline
%Method Calls    & 2     & 5.41\%    & no  \\ \hline
%Control Flow    & 2     & 5.41\%    & no  \\ \hline
%Call Stack      & 2     & 5.41\%    & no  \\ \hline
%??              & 2     & 5.41\%    & -   \\ \hline
%Tracing         & 1     & 2.70\%    & no  \\ \hline
%Recursion       & 1     & 2.70\%    & yes \\ \hline
%Parsing         & 1     & 2.70\%    & yes \\ \hline
%Objects         & 1     & 2.70\%    & no  \\ \hline
%Logic Operation & 1     & 2.70\%    & no  \\ \hline
%List            & 1     & 2.70\%    & yes \\ \hline
%HashSets        & 1     & 2.70\%    & no  \\ \hline
\hline
\textbf{Total} & 30    & 70.00\%   & -   \\
\hline
\end{tabular}
\caption{Notional machines in the dataset published by \citet{fincherNotionalMachinesComputing2020} classified according to their narrow focus.}
\label{tab:nm-classification-narrow-focus}
\end{table}


We then classified the notion of machines shown in Section~\ref{chr:Modeling} and Section~\ref{chr:RevealingInconsistencies} according to these dimensions.

\begin{table}[]
\begin{tabular}{|l||l|l|l|}
\hline
\textbf{Notional Machine}      & \textbf{Form} & \textbf{Broad Focus} & \textbf{Focused Construct} \\ \hline
\hline
\hline
Array a Row of Parkings Spaces & Analogy        & Evaluation     & Arrays       \\ \hline
List a Stack of Boxes          & Analogy        & Data Structure & List         \\ \hline
ExpTree                        & Representation & Evaluation     & -            \\ \hline
ExpTutorDiagram                & Representation & Evaluation     & -            \\ \hline
TAPLMemoryDiagram              & Representation & Evaluation     & References   \\ \hline
TypedExpTutorDiagram           & Representation & Type-checking  & -            \\ \hline
AlligatorEggs                  & Analogy        & Evaluation     & -            \\ \hline
\end{tabular}
\caption{Notional machines analyzed in this paper classified according to dare form, broad focus, and narrow focus.}
\label{tab:our-nm-classification}
\end{table}

\subsection{Expressing Complex Language Semantics}
\label{sec:more-complexity}

% - modeling layers
Although the examples presented here use only small languages,
the same approach can be used for larger languages.
We see
the dynamic and static conceptual models
to reason about Ownership Types in Rust~\citep{crichtonGroundedConceptualModel2023}
as such an example.
% fits nicely in our framework
The authors present two models: a dynamic and a static.
In our framework,
each model can be seen as a notional machine.
The authors divide each model into a formal model (about which formal statements can be made), which would correspond to the abstract representation of the notional machine,
and an informal model, which would correspond to the concrete representation of the notional machine.
%
In the dynamic model,
our PL layer would be the Rust language
and
our NM layer would be Miri.
%
In the static model,
our PL layer would be the Polonius model of borrow checking
and
our NM layer would be the \emph{permissions model} of borrow checking
(introduced by the authors).

