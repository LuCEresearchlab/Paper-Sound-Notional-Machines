\section{Designing Sound Notional Machines}
\label{chr:Modeling}


We can only begin to talk about the soundness of a notional machine if we have a
formal description of the programming language the notional machine is focused
on.
%
We use a set of small programming languages with well-known formalizations
described in
Pierce's Types and Programming Languages (TAPL) book~\citep{pierceTypesProgrammingLanguages2002}.
The languages are used to
explore different aspects of programming language semantics.
%
%Throughout this chapter,
%these languages are used
%in various examples
%that explore different aspects of the design of notional machines.
%
Table~\ref{tab:examples-designing-nms}
shows
the notional machines
we use in this section
as well as
the corresponding programming language
and aspect of the semantics of the programming language
that the notional machine focuses on.
%
We use the first example also to introduce the definition of soundness for notional machines.

We model each programming language and notional machine in Haskell.
The models are executable,
so they include implementations of the programming languages
(including parsers, interpreters, and type-checkers),
the notional machines,
and the relationship between them%
\footnote{%
The artifact containing a superset of the examples in this paper
is at
\textsc{Anonymous}.
%\url{https://github.com/LuCEresearchlab/sound-notional-machines}.
}.
The soundness proofs presented in this section are done using equational reasoning~\citep{gibbonsCalculatingFunctionalPrograms2002,birdAlgebraicIdentitiesProgram1989}.
%An implementation of the commutative diagrams of all the \nms{} presented in the thesis is available in an accompanying artifact.

\begin{table}[]
    \centering
    \begin{tabular}{|r||l|l|l|}
        \hline
        \textbf{Section}            & \textbf{Notional Machine}           & \textbf{Programming Language}    & \textbf{Focus}      \\ \hline\hline
        \ref{sec:ExpTree}  & \nmName{ExpTree}           & \plName{UntypedLambda}  & Evaluation \\ \hline
        \ref{sec:ExpTutor} & \nmName{ExpTutorDiagram}   & \plName{UntypedLambda}  & Evaluation \\ \hline
        % \ref{sec:Reduct} & \nmName{Reduct} & \plName{UntypedLambda} & step\\\hline
        % \ref{sec:AlligatorEggs} & \nmName{Alligator} & \plName{UntypedLambda} & step\\\hline
        \ref{sec:State}    & \nmName{TAPLMemoryDiagram} & \plName{TypedLambdaRef} & Evaluation (Refs) \\ \hline
        \ref{sec:Typing}   & \nmName{TypedExpTutorDiagram}   & \plName{TypedArith}     & Type-checking      \\ \hline
    \end{tabular}
    \caption{Notional machines, programming languages, and aspects of focus
    used in Section~\ref{chr:Modeling}.}
    \label{tab:examples-designing-nms}
\end{table}

\input{sections/BijectiveNM}

\input{sections/InjectiveNM}

\input{sections/TAPLMemoryDiagramNM}

\input{sections/TypedExpTutorNM}




%%%%%%%%%%%%%%%%%%%%%%%%%%%%%%%%%%%%%%
% \section{Conclusion}
% In this chapter, we have defined sound notional machines
% and shown how we can use the definition to devise a methodology to design notional machines that are sound by construction.
% We used the methodology to design
% different notional machines that
% focus on various aspects of
% multiple programming languages.
% In the next chapter, 
% we will use the definition to devise a methodology for analyzing existing notional machines.
