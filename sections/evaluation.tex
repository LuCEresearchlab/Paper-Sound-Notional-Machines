\section{Evaluation}
\label{chr:evaluation}

\begin{table}[]
\begin{tabular}{|l||r|l|l|}
\hline
\textbf{Form}  & \textbf{Count} & \textbf{Percentage} & \textbf{Covered} \\
\hline
\hline
Analogy        & 19    & 51.35\%    & yes \\ \hline
Representation & 17    & 45.95\%    & yes \\ \hline
Both           & 1     & 2.70\%     & -   \\ \hline
\hline
\textbf{Total} & 37    & 100.00\%   & -   \\
\hline
\end{tabular}
\caption{Notional machines in the dataset published by \citet{fincherNotionalMachinesComputing2020} classified according to their form.}
\label{tab:nm-classification-form}
\end{table}

\begin{table}[]
\begin{tabular}{|l||r|l|l|}
\hline
\textbf{Broad Focus} & \textbf{Count} & \textbf{Percentage} & \textbf{Covered} \\
\hline
\hline
Eval Construct & 27    & 72.97\%   & yes \\ \hline
Data Structure & 3     & 8.11\%    & yes \\ \hline
Type-checking  & 2     & 5.41\%    & yes \\ \hline
Parsing Prog   & 2     & 5.41\%    & yes \\ \hline
??             & 2     & 5.41\%    & -   \\ \hline
Logic Gates    & 1     & 2.70\%    & no  \\ \hline
\hline
\textbf{Total} & 37    & 100.00\%  & -   \\
\hline
\end{tabular}
\caption{Notional machines in the dataset published by \citet{fincherNotionalMachinesComputing2020} classified according to their broad focus.}
\label{tab:nm-classification-broad-focus}
\end{table}

\begin{table}[]
\begin{tabular}{|l||r|l|l|}
\hline
\textbf{Narrow Focus} & \textbf{Count} & \textbf{Percentage} & \textbf{Covered} \\
\hline
\hline
Memory          & 7     & 18.92\%   & yes \\ \hline
Variables       & 5     & 13.51\%   & yes \\ \hline
Arrays          & 4     & 10.81\%   & yes \\ \hline
Functions       & 3     & 8.11\%    & yes \\ \hline
Expression      & 3     & 8.11\%    & yes \\ \hline
Method Calls    & 2     & 5.41\%    & no  \\ \hline
Control Flow    & 2     & 5.41\%    & no  \\ \hline
Call Stack      & 2     & 5.41\%    & no  \\ \hline
??              & 2     & 5.41\%    & -   \\ \hline
Tracing         & 1     & 2.70\%    & no  \\ \hline
Recursion       & 1     & 2.70\%    & yes \\ \hline
Parsing         & 1     & 2.70\%    & yes \\ \hline
Objects         & 1     & 2.70\%    & no  \\ \hline
Logic Operation & 1     & 2.70\%    & no  \\ \hline
List            & 1     & 2.70\%    & yes \\ \hline
HashSets        & 1     & 2.70\%    & no  \\ \hline
\hline
\textbf{Total} & 37    & 100.00\%   & -   \\
\hline
\end{tabular}
\caption{Notional machines in the dataset published by \citet{fincherNotionalMachinesComputing2020} classified according to their narrow focus.}
\label{tab:nm-classification-narrow-focus}
\end{table}

\begin{table}[]
\begin{tabular}{|l||l|l|l|}
\hline
\textbf{Notional Machine}      & \textbf{Form} & \textbf{Broad Focus} & \textbf{Narrow Focus} \\ \hline
\hline
\hline
Array a Row of Parkings Spaces & Analogy        & Eval Construct & Arrays       \\ \hline
List a Stack of Boxes          & Analogy        & Data Structure & List         \\ \hline
ExpTree                        & Representation & Eval Construct & -            \\ \hline
ExpTutorDiagram                & Representation & Eval Construct & -            \\ \hline
TAPLMemoryDiagram              & Representation & Eval Construct & References   \\ \hline
TypedExpTutorDiagram           & Representation & Type-checking  & -            \\ \hline
AlligatorEggs                  & Analogy        & Eval Construct & -            \\ \hline
\end{tabular}
\caption{Notional machines analyzed in this paper classified according to dare form, broad focus, and narrow focus.}
\label{tab:our-nm-classification}
\end{table}

\subsection{Expressing Complex Language Semantics}
\label{sec:more-complexity}

% - modeling layers
Although the examples presented here use only small languages,
the same approach can be used for larger languages.
We see
the dynamic and static conceptual models
to reason about Ownership Types in Rust~\citep{crichtonGroundedConceptualModel2023}
as such an example.
% fits nicely in our framework
The authors present two models: a dynamic and a static.
In our framework,
each model can be seen as a notional machine.
The authors divide each model into a formal model (about which formal statements can be made), which would correspond to the abstract representation of the notional machine,
and an informal model, which would correspond to the concrete representation of the notional machine.
%
In the dynamic model,
our PL layer would be the Rust language
and
our NM layer would be Miri.
%
In the static model,
our PL layer would be the Polonius model of borrow checking
and
our NM layer would be the \emph{permissions model} of borrow checking
(introduced by the authors).

