%%%%%%%%%%%%%%%%%%%%%%%%%%%%%%%%%%%%%%
\section{Conclusions}
\label{sec:Conclusions}

% - problem
A notional machine
%is a pedagogic device to assist the understanding of some aspect of programs or programming
%under focus by the notional machine.
%%they are popular
%They
are popular in computer science education,
commonly used both by instructors in their teaching practice
as well as by researchers.
%
Despite
their popularity,
there has been no precise formal characterization of
what should be the relationship between
a notional machine
and the aspect under its focus
%
that would allow one to
evaluate whether or not they are consistent with each other.
%
%
% - solution
% -- definition
We, therefore, introduced
a definition
of soundness for notional machines.
The definition is based on
%the idea of
simulation,
a
well-established notion
widely used in many areas of computer science.
%to proofs of correctness of data representations.
%
Demonstrating soundness essentially
amounts to
constructing
a commutative diagram relating the \nm{}
with the object of its focus.


% -- methodologies
Using this definition,
we
showed how we can
(1)
systematically
design notional machines that are sound by construction,
%
%which
%we demonstrate
%with a series of examples
%that explore different
%notional machines
%and
%aspects of programming language semantics,
and
(2)
analyze existing notional machines
to uncover inconsistencies
and suggest improvements.


% -- abstract versus concrete
An important insight
in the process
is to
distinguish between
the concrete representation of the notional machine
(typically visual)
and its abstract representation,
about which we can make formal statements.
%
This distinction is akin to the distinction between
the concrete and the abstract syntaxes of a programming language.
 




% - takeaways
% -- reasoning about notional machines
This work intends more generally to establish a
%The approach presented here is meant more broadly as a general
framework to reason about notional machines,
placing the research on notional machines on firmer ground.
As such,
it can be
used
to address challenges
such as the design, analysis, and evaluation of notional machines,
and
the construction of automated tools based on notional machines.


